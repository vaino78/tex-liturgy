
\kinovarsimple{Чи'нъ свяще'нныя и= бж~е'ственныя лiтургi'и ст~а'гw
i=wа'нна златоу'стагw.}

\ruber{Хотя'й свяще'нникъ бж~е'ственное соверша'ти 
тайнодjь'йствiе, до'л\-женъ _е='сть пе'рвjье о_у='бw 
примире'нъ бы'ти со всjь'ми, и= не и=мjь'ти что` на 
кого`, и= се'рдце же, _е=ли'ка си'ла, w\т лука'выхъ 
блюсти` по'мыслwвъ, воздержа'тися же съ ве'чера, и= 
трезви'тися да'же до вре'мене священнодjь'йствiя.}

\ruber{Вре'мени же наста'вшу, повнегда` соверши'ти _о=бы'чный 
предстоя'телю покло'нъ, вхо'дитъ въ хра'мъ: и= 
соедини'вся со дiа'кономъ, творя'тъ вку'пjь къ восто'ку 
пред\ъ ст~ы'ми две'рьми поклон_е'нiя три`.}

\ruberc{Та'же гл~го'летъ дiа'конъ:}

\kinovar{Б}{л~гослови`} вл\дко.

\ruberc{Сщ~е'нникъ:}

\kinovar{Б}{л~гослове'нъ} бг~ъ на'шъ, всегда`, ны'нjь и= при'снw, 
и= во вjь'ки вjькw'въ, а=ми'нь.

\ruberc{Начина'етъ глаго'лати дiа'конъ:}

\kinovar{Ц}{р~ю`} нб\сный: \kinovarsimple{Трист~о'е.} \kinovarsimple{По} \kinovar{_О='}{ч~е} на'шъ:

\ruber{Сщ~е'нник\ъ:} \kinovar{Jа='}{кw} твое` _е='сть цр\ство:

\ruberc{Та'же глаго'лютъ:}

\kinovar{П}{оми'луй} на'съ гд\си, поми'луй на'с\ъ: вся'кагw бо 
w\твjь'та недоумjь'юще, сiю' ти мл~тву jа='кw вл\дцjь 
грjь'шнiи прино'симъ: поми'луй на'съ.

\kinovarsimple{Сла'ва:} \kinovar{Г}{д\си,} поми'луй на'съ, на тя' бо 
о_у=пова'хомъ, не прогнjь'вайся на ны` sjьлw`, ниже` 
помяни` беззако'нiй на'шихъ: но при'зри и= ны'нjь jа='кw 
бл~гоутро'бенъ, и= и=зба'ви ны` w\т врагw'въ на'шихъ. ты' 
бо _е=си` бг~ъ на'шъ, и= мы` лю'дiе твои`, вси` дjьла` 
руку` твое'ю, и= и='мя твое` призыва'емъ.

\kinovarsimple{И=} \kinovarsimple{ны'нjь:} \kinovar{М}{илосе'рдiя} дв_е'ри w\тве'рзи на'мъ 
бл~гослове'нная бц\де, надjь'ющiися на тя` да не 
поги'бнемъ, но да и=зба'вимся тобо'ю w\т бjь'дъ: ты' бо 
_е=си` спасе'нiе ро'да хрiстiа'нскагw.

\ruber{Та'же w\тхо'дятъ ко i=кw'нjь хр\сто'вjь, глаго'люще:}

\kinovar{П}{реч\стому} твоему` _о='бразу покланя'емся благi'й, 
прося'ще проще'нiя прегрjьше'нiй на'шихъ хр\сте` бж~е: 
во'лею бо благоволи'лъ _е=си` пло'тiю взы'ти на кр\стъ, 
да и=зба'виши, jа=`же созда'лъ _е=си`, w\т рабо'ты 
вра'жiя. тjь'мже благода'рственнw вопiе'м\ъ ти`: ра'дости 
и=спо'лнилъ _е=си` вся^ сп~се на'шъ, прише'дый спасти` 
мi'ръ.

\ruber{Та'же цjьлу'ютъ и= i=кw'ну бц\ды, глаго'люще тропа'рь:}

\kinovar{М}{илосе'рдiя} су'щи и=сто'чник\ъ, ми'лости сподо'би на'съ 
бц\де, при'зри на лю'ди согрjьши'вшыя, jа=ви` jа='кw 
при'снw си'лу твою`: на тя' бо о_у=пова'юще, ра'дуйся, 
вопiе'мъ ти`, jа='кw и=ногда` гаврiи'лъ безпло'тныхъ 
а=рхiстрати'гъ.

\ruberc{Та'же прикло'нь главу`, i=ере'й глаго'летъ:}

\kinovar{Г}{д\си,} низпосли` ру'ку твою` съ высоты` ст~а'гw 
жили'ща твоегw`, и= о_у=крjьпи` мя` въ предлежа'щую 
слу'жбу твою`: да неwсужде'ннw предста'ну стра'шному 
пр\сто'лу твоему`, и= безкро'вное священнодjь'йствiе 
совершу`, jа='кw твоя` _е='сть си'ла, во вjь'ки вjькw'въ, 
а=ми'нь.

\ruber{Та'же творя'тъ и= къ ликw'мъ покло'ны по _е=ди'ному, 
и= та'кw w\тхо'дятъ въ же'ртвенникъ, глаго'люще:}

\kinovar{В}{ни'ду} въ до'мъ тво'й, поклоню'ся ко хра'му ст~о'му 
твоему` въ стра'сjь твое'мъ. гд\си, наста'ви мя` пра'вдою 
твое'ю, вра^гъ мои'хъ ра'ди и=спра'ви пред\ъ тобо'ю пу'ть 
мо'й: jа='кw нjь'сть во о_у=стjь'хъ и='хъ и='стины, 
се'рдце и='хъ су'етно, гро'бъ w\тве'рстъ горта'нь и='хъ, 
я=зы'ки свои'ми льща'ху. суди` и=`мъ бж~е, да w\тпаду'тъ 
w\т мы'слей свои'хъ, по мно'жеству нече'стiя и='хъ 
и=зри'ни я=`, jа='кw преwгорчи'ша тя` гд\си. и= да 
возвеселя'тся вси` о_у=пова'ющiи на тя`, во вjь'ки 
возра'дуются, и= всели'шися въ ни'хъ, и= похва'лятся w= 
тебjь` лю'бящiи и='мя твое`. jа='кw ты` благослови'ши 
пра'ведника гд\си, jа='кw _о=ру'жiемъ благоволе'нiя 
вjьнча'лъ _е=си` на'съ.

\ruber{Вше'дше же во святи'лище, творя'тъ покло'ны три` 
пред\ъ ст~о'ю трапе'зою, и= цjьлу'ютъ ст~о'е _е=v\глiе и= 
ст~у'ю трапе'зу. Та'же прiе'млютъ въ ру'ки своя^ кi'йждо 
стiха'рь сво'й, и= творя'тъ покло'ны три` къ восто'ку, 
глаго'люще въ себjь` кi'йждо:}

\kinovar{Б}{ж~е,} w=чи'сти мя` грjь'шнаго и= поми'луй мя`.

\ruber{Та'же прихо'дитъ къ свяще'ннику дiа'конъ, держя` въ 
десно'й руцjь` стiха'рь со _о=раре'мъ, и= подклони'въ 
_е=му` главу`, глаго'летъ:}

\kinovar{Б}{л~гослови`,} вл\дко, стiха'рь со _о=раре'мъ.

\ruberc{Сщ~е'нникъ глаго'летъ:}

\kinovar{Б}{л~гослове'нъ} бг~ъ на'шъ, всегда`, ны'нjь и= при'снw, 
и= во вjь'ки вjькw'въ.

\ruber{Та'же w\тхо'дитъ къ себjь` дiа'конъ, во _е=ди'ну 
страну` святи'лища, и= w=блачи'тся въ стiха'рь, моля'ся 
си'це:}

\kinovar{В}{озра'дуется} душа` моя` w= гд\сjь, w=блече' бо мя` въ 
ри'зу спасе'нiя, и= w=де'ждею весе'лiя w=дjь'я мя`: 
jа='кw жениху` возложи` ми` вjьне'цъ, и= jа='кw невjь'сту 
о_у=краси' мя красото'ю.

\ruber{И= _о=ра'рь о_у='бw цjьлова'въ, налага'етъ на лjь'вое 
ра'мо. Нарука^вницы же налага'я на ру'ки, на десну'ю о_у='бw, 
глаго'летъ:}

\kinovar{Д}{есни'ца} твоя`, гд\си, просла'вися въ крjь'пости: 
десна'я твоя` рука`, гд\си, сокруши` враги`, и= 
мно'жествомъ сла'вы твоея` сте'рлъ _е=си` супоста'ты.

\ruber{На лjь'вую же, глаго'летъ:} \kinovar{Р}{у'цjь} твои` сотвори'стjь 
мя` и= созда'стjь мя`. вразуми' мя, и= научу'ся 
за'повjьдемъ твои^мъ.

\ruber{Та'же w\тше'дъ в\ъ предложе'нiе, о_у=готовля'етъ 
сщ~_е'нная. Ст~ы'й о_у='бw дi'скосъ поставля'етъ w= шу'юю 
страну`, поти'рь же, _е='же _е='сть ст~у'ю ча'шу, w= 
десну'ю, и= прw'чая съ ни'ми.}

\ruber{Сщ~е'нникъ же си'це 
w=блачи'тся: прiе'мъ стiха'рь въ лjь'вую ру'ку, и= 
покло'нився три'жды къ восто'ку, jа='коже рече'ся, 
назна'менуетъ глаго'ля:}

\kinovar{Б}{л~гослове'нъ} бг~ъ на'шъ, всегда`, ны'нjь и= при'снw, 
и= во вjь'ки вjькw'въ, а=ми'нь.

\ruber{Та'же w=блачи'тся, глаго'ля:} 
\kinovar{В}{о}з\-ра'\-ду\-ет\-ся душа` моя` 
w= гд\сjь: \ruber{до конца`.}

\ruber{Та'же прiе'мъ _е=пiтрахи'ль, и= назна'менавъ, 
w=блача'ется _е='ю, глаго'ля:}

\kinovar{Б}{л~гослове'нъ} бг~ъ, и=злива'яй бл~года'ть свою` на 
свяще'нники своя^: jа='кw мv'ро на главjь`, сходя'щее на 
браду`, браду` а=арw'ню, сходя'щее на _о=ме'ты w=де'жды 
_е=гw`.

\ruber{Та'же прiе'мъ поя'съ и= w=поясу'яся, глаго'летъ:}

\kinovar{Б}{л~гослове'нъ} бг~ъ, препоясу'яй мя` си'лою, и= положи` 
непоро'ченъ пу'ть мо'й, соверша'яй но'зjь мои` jа='кw 
_е=ле'ни, и= на высо'кихъ поставля'яй мя`.

\ruber{Нарука^вницы же, jа='кw вы'ше рече'ся.}

\ruber{Та'же прiе'мъ набе'дренникъ, а='ще и='мать, 
и= благослови'въ и=`, и= цjьлова'въ, глаго'летъ:}

\kinovar{П}{репоя'ши} ме'чь тво'й по бедрjь` твое'й си'льне, 
красото'ю твое'ю, и= добро'тою твое'ю, и= наляцы`, и= 
о_у=спjьва'й и= ца'рствуй, и='стины ра'ди и= кро'тости и= 
пра'вды, и= наста'витъ тя` ди'внw десни'ца твоя`, 
всегда`, ны'нjь и= при'снw, и= во вjь'ки вjькw'въ, 
а=ми'нь.

\ruber{Та'же прiе'мъ фелw'нь и= бл~гослови'въ, цjьлу'етъ гл~я 
си'це:}

\kinovar{С}{щ~е'нницы} твои`, гд\си, w=блеку'тся въ пра'вду, и= 
преподо'бнiи твои` ра'достiю возра'дуются, всегда`, 
ны'нjь и= при'снw, и= во вjь'ки вjькw'въ, а=ми'нь.

\ruber{Та'же w\тше'дше въ предложе'нiе, о_у=мыва'ютъ ру'ки, 
глаго'люще:}

\kinovar{О_у=}{мы'ю} въ непови'нныхъ ру'цjь мои`, и= w=бы'ду 
же'ртвенникъ тво'й гд\си, _е='же о_у=слы'шати ми` гла'съ 
хвалы` твоея`, и= повjь'дати вся^ чудеса` твоя^. гд\си, 
возлюби'хъ бл~голjь'пiе до'му твоегw`, и= мjь'сто 
селе'нiя сла'вы твоея`: да не погуби'ши съ нечести'выми 
ду'шу мою`, и= съ му'жи крове'й живота` моегw`, и='хже въ 
рука'хъ беззакw'нiя, десни'ца и='хъ и=спо'лнися мзды`. 
А='зъ же неsло'бiемъ мои'мъ ходи'хъ, и=зба'ви мя` гд\си, 
и= поми'луй мя`: нога` моя` ста` на правотjь`, въ 
це'рквахъ благословлю` тя` гд\си.

\ruberc{И= та'кw w\тхо'дятъ въ предложе'нiе.}

\ruber{Та'же поклон_е'нiя три` пред\ъ предложе'нiемъ 
сотво'рше, глаго'лютъ кi'й\-ждо:}

\kinovar{Б}{ж~е,} w=чи'сти мя` грjь'шнаго, и= поми'луй мя`. \ruber{и=} \kinovar{И=}{скупи'лъ} ны` _е=си` w\т кля'твы зако'нныя ч\стно'ю 
твое'ю кро'вiю, на кр\стjь` пригвозди'вся и= копiе'мъ 
пробо'дся, безсме'ртiе и=сточи'лъ _е=си` человjь'кwмъ: 
сп~се на'шъ, сла'ва тебjь`.

\ruberc{Та'же глаго'летъ дiа'конъ:}

\kinovar{Б}{л~гослови`} вл\дко.

\ruberc{И= начина'етъ сщ~е'нникъ:}

\kinovar{Б}{л~}го\-сло\-ве'нъ бг~ъ на'шъ, 
всегда`, ны'нjь и= при'снw, и= во вjь'ки вjькw'въ.

\ruber{Дiа'конъ:} \kinovar{А=}{ми'нь.}

\kinovarsimple{Та'же} \kinovarsimple{прiе'млетъ} \kinovarsimple{сщ~е'нникъ} 
\kinovarsimple{лjь'вою} \kinovarsimple{о_у='бw} \kinovarsimple{руко'ю} 
\kinovarsimple{просфору`,} \kinovarsimple{десно'ю} \kinovarsimple{же} 
\kinovarsimple{ст~о'е} \kinovarsimple{копiе`,} \kinovarsimple{и=} 
\kinovarsimple{зна'менуяй} \kinovarsimple{съ} \kinovarsimple{ни'мъ} 
\kinovarsimple{три'жды} \kinovarsimple{верху`} \kinovarsimple{печа'ти} 
\kinovarsimple{просфоры`,} \kinovarsimple{глаго'летъ:} \kinovar{В}{ъ} 
воспомина'нiе гд\са, и= бг~а, и= сп~са на'шегw i=и~са 
хр\ста`. \kinovarsimple{[Три'жды.]}

\kinovarsimple{И=} \kinovarsimple{а='бiе} \kinovarsimple{водружа'етъ} \kinovarsimple{копiе`} \kinovarsimple{въ} \kinovarsimple{десну'ю} \kinovarsimple{страну`} \kinovarsimple{печа'ти,} \kinovarsimple{и=} \kinovarsimple{глаго'летъ} \kinovarsimple{рjь'жя:} \kinovar{Jа='}{кw} _о=вча` на 
заколе'нiе веде'ся. \kinovarsimple{Въ лjь'вую же: И=} jа='кw а='гнецъ 
непоро'ченъ пря'мw стригу'щагw _е=го` безгла'сенъ, та'кw 
не w\тверза'етъ о_у='стъ свои'хъ. \kinovarsimple{Въ} \kinovarsimple{го'рнюю} \kinovarsimple{же} \kinovarsimple{страну`} \kinovarsimple{печа'ти:} \kinovar{В}{о} смире'нiи _е=гw` су'дъ _е=гw` взя'тся. \kinovarsimple{Въ} \kinovarsimple{до'льнюю} \kinovarsimple{же} \kinovarsimple{страну`:} \kinovar{Р}{о'дъ} же _е=гw` кто` и=сповjь'сть; 
\kinovarsimple{Дiа'конъ} \kinovarsimple{же,} \kinovarsimple{взира'я} \kinovarsimple{бл~го\-го\-вjь'й\-нw} \kinovarsimple{на} \kinovarsimple{сицево`} \kinovarsimple{та'инство,} \kinovarsimple{глаго'летъ} \kinovarsimple{на} \kinovarsimple{_е=ди'номъ} \kinovarsimple{ко'емждо} \kinovarsimple{рjь'занiи:} \kinovar{Г}{д\су} 
помо'лимся. \kinovarsimple{держя`} \kinovarsimple{и=} \kinovarsimple{_о=ра'рь} \kinovarsimple{въ} \kinovarsimple{руцjь`.} \kinovarsimple{По} \kinovarsimple{си'хъ} \kinovarsimple{глаго'летъ:} \kinovar{В}{озми`,} вл\дко. \kinovarsimple{Сщ~е'нникъ} \kinovarsimple{же,} \kinovarsimple{вложи'въ} \kinovarsimple{ст~о'е} \kinovarsimple{копiе`} \kinovarsimple{w\т} \kinovarsimple{ко'свенныя} \kinovarsimple{десны'я} \kinovarsimple{страны`} \kinovarsimple{просфоры`,} \kinovarsimple{взима'етъ} \kinovarsimple{ст~ы'й} \kinovarsimple{хлjь'бъ,} \kinovarsimple{глаго'ля} \kinovarsimple{си'це:} \kinovar{Jа='}{кw} 
взе'млется w\т земли` живо'тъ _е=гw`. \kinovarsimple{И=} \kinovarsimple{положи'въ} \kinovarsimple{и=} \kinovarsimple{взна'къ} \kinovarsimple{на} \kinovarsimple{ст~jь'мъ} \kinovarsimple{дi'скосjь,} \kinovarsimple{ре'кшу} \kinovarsimple{дiа'кону:} \kinovar{П}{ожри`,} 
вл\дко. \kinovarsimple{жре'тъ} \kinovarsimple{_е=го`} \kinovarsimple{кр\стови'днw,} \kinovarsimple{си'це} \kinovarsimple{глаго'ля:} \kinovar{Ж}{ре'тся} а='гнецъ бж~iй, взе'мляй грjь'хъ мi'ра, за 
мiрскi'й живо'тъ и= спасе'нiе.

\kinovarsimple{И=} \kinovarsimple{w=браща'етъ} \kinovarsimple{другу'ю} \kinovarsimple{страну`} 
\kinovarsimple{горjь`,} \kinovarsimple{и=му'щую} \kinovarsimple{кр\стъ.} 
\kinovarsimple{И=} \kinovarsimple{гл~етъ} \kinovarsimple{дiа'конъ:} \kinovar{П}{рободи`,} вл\дко. \kinovarsimple{I=ере'й 
же пробода'я и=` въ десну'ю страну` ст~ы'мъ копiе'мъ, 
глаго'летъ:}

\kinovar{_Е=}{ди'нъ} w\т вw'инъ копiе'мъ ре'бра _е=гw` прободе`, 
и= а='бiе и=зы'де кро'вь и= вода`: и= ви'дjьвый 
свидjь'тельствова, и= и='стинно _е='сть свидjь'тельство 
_е=гw`.

\kinovarsimple{Дiа'конъ} \kinovarsimple{же} \kinovarsimple{влива'етъ} \kinovarsimple{во} \kinovarsimple{ст~ы'й} \kinovarsimple{поти'рь} \kinovarsimple{w\т} \kinovarsimple{вiна`} \kinovarsimple{вку'пjь} \kinovarsimple{и=} \kinovarsimple{воды`,} \kinovarsimple{рекi'й} \kinovarsimple{пре'жде} \kinovarsimple{ко} \kinovarsimple{сщ~е'ннику:} \kinovar{Б}{л~гослови`,} вл\дко, ст~о'е соедине'нiе. и= взе'мъ над\ъ 
ни'ми бл~гослове'нiе.

[\kinovar{в~}{]} \kinovarsimple{Сщ~е'нникъ же прiе'мъ въ ру'цjь втору'ю 
просфору`, глаго'летъ:}

\kinovar{В}{ъ} че'сть и= па'мять пребл~гослове'нныя вл\дчцы на'шея 
бц\ды и= приснодв~ы мр~i'и, _е=я'же моли'твами прiими`, 
гд\си, же'ртву сiю` въ пренб\сный тво'й же'ртвенникъ.

\kinovarsimple{И= взе'мъ ча'стицу, полага'етъ ю=` w=десну'ю ст~а'гw 
хлjь'ба, бли'зъ среды` _е=гw`, глаго'ля:}

\kinovar{П}{редста`} цр~и'ца w=десну'ю тебе`, въ ри^зы позлаще'нны 
w=дjь'яна, преукраше'нна.

[\kinovar{г~}{]} \kinovarsimple{Та'же прiе'мъ тре'тiю просфору`, глаго'летъ:}

[а~] \kinovar{Ч}{естна'гw} сла'внагw проро'ка, пр\дте'чи и= 
кр\сти'теля i=wа'нна.

\kinovarsimple{И= взе'мъ пе'рвую ча'стицу, полага'етъ ю=` w= лjь'вую 
страну` ст~а'гw хлjь'ба, творя'й нача'ло пе'рвагw чи'на.}

\kinovarsimple{Та'же глаго'летъ:}

[в~] \kinovar{С}{т~ы'хъ} сла'вныхъ пр\оро'кwвъ: мwv"се'а и= 
а=арw'на, и=лiи` и= _е=лiссе'а, дв~да и= i=ессе'а: 
ст~ы'хъ трiе'хъ _о=трокw'въ, и= данiи'ла проро'ка, и= 
всjь'хъ ст~ы'хъ проро'кwвъ.

\kinovarsimple{И= взе'мъ ча'стицу, полага'етъ ю=` до'лjь пе'рвыя 
благочи'ннw.}

\kinovarsimple{Та'же па'ки глаго'летъ:}

[г~] \kinovar{С}{т~ы'хъ} сла'вныхъ и= всехва'льныхъ а=п\слъ петра` 
и= па'vла, и= про'чихъ всjь'хъ ст~ы'хъ а=п\столwвъ.

\kinovarsimple{И= та'кw полага'етъ тре'тiю ча'стицу до'лjь вторы'я, 
скончава'я пе'рвый чи'нъ.}

\kinovarsimple{Та'же глаго'летъ:}

[а~] \kinovar{И=`}{же} во ст~ы'хъ _о=ц~ъ на'шихъ ст~и'телей, 
васi'лiа вели'кагw, григо'рiа бг~осло'ва, и= i=wа'нна 
златоу'стагw, а=fана'сiа и= кv"рi'лла, нiкола'а 
мv"рлv"кi'йскагw, меfо'дiа о_у=чи'теля слове'нскагw, 
мiхаи'ла кi'евскагw, петра`, и= а=ле_ксi'а, и= i=w'ны, и= 
фiлi'ппа моско'вскихъ, нiки'ты _е=пi'скопа 
новгоро'дскагw, лео'нтiа _е=пi'скопа росто'вскагw, и= 
всjь'хъ ст~ы'хъ святи'телей.

\kinovarsimple{И= взе'мъ четве'ртую ча'стицу, полага'етъ ю=` бли'зъ 
пе'рвыя ча'стицы, творя` второ'е нача'ло.}

\kinovarsimple{Та'же па'ки глаго'летъ:}

[в~] \kinovar{С}{т~а'гw} а=по'стола, первому'ченика и= 
а=рхiдiа'кона стефа'на, ст~ы'хъ вели'кихъ мч~нкwвъ, 
дими'трiа, геw'ргiа, fео'дwра тv'рwна, fео'дwра 
стратила'та, и= всjь'хъ ст~ы'хъ му'ч_еникъ: и= мч~нцъ, 
fе'клы, варва'ры, кv"рiакi'и, _е=vфи'мiи и= параске'vы, 
_е=катерi'ны, и= всjь'хъ ст~ы'хъ му'ченицъ.

\kinovarsimple{И= взе'мъ пя'тую ча'стицу, полага'етъ ю=` до'лjь 
пе'рвыя, су'щiя нача'ломъ втора'гw чи'на.}

\kinovarsimple{Та'же глаго'летъ:}

[г~] \kinovar{П}{реподо'бныхъ} и= бг~оно'сныхъ _о=ц~ъ на'шихъ, 
а=нтw'нiа, _е=vfv'мiа, са'ввы, _о=ну'фрiа, а=fана'сiа 
а=fw'нскагw, кv"рi'лла о_у=чи'теля слове'нскагw, 
а=нтw'нiа и= fеодо'сiа пече'рскихъ, се'ргiа 
ра'донежскагw, варлаа'ма хуты'нскагw, и= всjь'хъ 
прп\дбныхъ _о=т_е'цъ: и= прп\дбныхъ ма'терей, пелагi'и, 
fеодо'сiи, а=настасi'и, _е=vпра_ксi'и, феvрw'нiи, 
fеоду'лiи, _е=vфросv'нiи, марi'и _е=гv'птяныни, и= 
всjь'хъ ст~ы'хъ преп\дбныхъ ма'терей.

\kinovarsimple{И= та'кw взе'мъ шесту'ю ча'стицу, полага'етъ ю=` 
до'лjь вторы'я ча'стицы, во и=сполне'нiе втора'гw чи'на.}

\kinovarsimple{По си'хъ же глаго'летъ:}

[а~] \kinovar{С}{т~ы'хъ} и= чудотво'рц_евъ безсре'бр_еникъ, космы` 
и= дамiа'на, кv'ра и= i=wа'нна, пантелеи'мона и= 
_е=рмола'а, и= всjь'хъ ст~ы'хъ безсре'бреникwвъ.

\kinovarsimple{И= взе'мъ седму'ю ча'стицу, полага'етъ ю=` ве'рхъ, 
творя` тре'тiе нача'ло, по чи'ну.}

\kinovarsimple{Та'же па'ки глаго'летъ:}

[в~] \kinovar{С}{т~ы'хъ} и= пра'ведныхъ бг~о_оц~ъ, i=wакi'ма и= 
а='нны, и= ст~а'гw, и='м\ркъ, _е=гw'же _е='сть де'нь, и= 
всjь'хъ ст~ы'хъ, и='хже моли'твами посjьти' ны, бж~е.

\kinovarsimple{И= полага'етъ _о=сму'ю ча'стицу до'лjь пе'рвыя 
благочи'ннw.}

\kinovarsimple{_Е=ще' же къ си^мъ глаго'летъ:}

[г~] \kinovar{И='}{же} во ст~ы'хъ _о=ц~а` на'шегw i=wа'нна, 
а=рхiеп\скпа кwнстантiнопо'льскагw, златоу'стагw. \kinovarsimple{[А='ще 
пое'тся лiтургi'а _е=гw`: а='ще же пое'тся вели'кагw 
васi'лiа, того` помина'етъ.]}

\kinovarsimple{И= та'кw взе'мъ f~-ю ча'стицу, полага'етъ ю=` въ 
коне'цъ тре'тiягw чи'на во и=сполне'нiе.}

[\kinovar{д~}{]} \kinovarsimple{Та'же прiе'мъ четве'ртую просфору`, 
глаго'летъ:}

\kinovar{П}{омяни`,} вл\дко человjьколю'бче, 
правосла'вное _е=п\скопство гони'мыя це'ркве рwссi'йскiя, 
господи'на на'шегw высокопреwсвяще'ннjьйшагw митрополi'та 
\kinovar{и='мярекъ}{,} первоiера'рха ру'сскiя правосла'вныя 
зарубе'жныя це'ркве, господи'на на'шегw преосвяще'ннjьйшагw 
а=рхiепi'скопа \kinovarsimple{[и=ли`} _е=пi'скопа\kinovarsimple{]} \kinovarsimple{и='мярекъ,} \kinovarsimple{_е=гw'же} \kinovarsimple{_е='сть} \kinovar{_е=па'рхiа}{,} и= вся'кое _е=п\скопство 
правосла'вныхъ, честно'е пресвv'терство, 
во хр\стjь` дiа'конство, и= ве'сь свяще'нническiй чи'нъ: 
\kinovarsimple{[а='ще во _о='би'тели:} и=гу'мена \kinovarsimple{и=ли`} а=рхiмандрi'та, 
\kinovarsimple{и='м\ркъ,]} бра'тiю и= сослуже'бники на'шя, 
свяще'нники, дiа'коны, и= всю` бра'тiю на'шу, jа=`же 
призва'лъ _е=си` во твое` _о=бще'нiе, твои'мъ 
благоутро'бiемъ, всеблагi'й вл\дко.

\kinovarsimple{И= взе'мъ ча'стицу, полага'етъ ю=` до'лjь ст~а'гw 
хлjь'ба.}

\kinovarsimple{Та'же помина'етъ страну` на'шу, глаго'ля си'це:}

\kinovar{П}{омяни`,} гд\си, стра'ждущую страну` на'шу рwссi'йскую 
и= правосла'вныхъ люде'й _е=я` во _о=те'чествjь и= 
разсjь'янiи су'щихъ.

\kinovarsimple{Та'же} \kinovarsimple{помина'етъ} \kinovarsimple{и=} \kinovarsimple{и=`хже} \kinovarsimple{и='мать} \kinovarsimple{живы'хъ} \kinovarsimple{по} \kinovarsimple{и='мени,} \kinovarsimple{и=} \kinovarsimple{на} \kinovarsimple{ко'еждо} \kinovarsimple{и='мя} \kinovarsimple{взима'етъ} \kinovarsimple{ча'стицу,} \kinovarsimple{приглаго'ля:} \kinovar{П}{омяни`,} гд\си, и='м\ркъ.

\kinovarsimple{И= та'кw взе'мъ ча^стицы, полага'етъ я=` до'лjь 
ст~а'гw хлjь'ба.}

[\kinovar{_е~}{]} \kinovarsimple{Та'же взе'мъ пя'тую просфору`, глаго'летъ:}

\kinovarsimple{W=} па'мяти и= w=ставле'нiи грjьхw'въ ст~jь'йшихъ 
патрiа'рхwвъ, правосла'вныхъ и= благочести'выхъ цр~е'й и= 
благочести'выхъ цр~и'цъ, бл~же'нныхъ созда'телей ст~ы'я 
_о=би'тели сея`.

\kinovarsimple{Та'же} \kinovarsimple{помина'етъ} \kinovarsimple{рукоположи'вшаго} \kinovarsimple{_е=го`} \kinovarsimple{а=рхiере'а,} \kinovarsimple{и=} \kinovarsimple{други'хъ,} \kinovarsimple{и=`хже} \kinovarsimple{хо'щетъ,} \kinovarsimple{о_у=со'пшихъ} \kinovarsimple{по} \kinovarsimple{и='мени.} \kinovarsimple{На} \kinovarsimple{ко'еждо} \kinovarsimple{и='мя} \kinovarsimple{взима'етъ} \kinovarsimple{ча'стицу,} \kinovarsimple{приглаго'ля:} \kinovar{П}{омяни`,} 
гд\си, и='м\ркъ.

\kinovarsimple{И= коне'чнjь глаго'летъ си'це:}

\kinovarsimple{И=} всjь'хъ въ наде'ждjь воскр\снiя, жи'зни вjь'чныя и= 
твоегw` _о=бще'нiя о_у=со'пшихъ, правосла'вныхъ _о=т_е'цъ 
и= бра'тiй на'шихъ, чл~вjьколю'бче гд\си. И= взима'етъ 
ча'стицу.

\kinovarsimple{Посе'мъ} \kinovarsimple{глаго'летъ:} \kinovar{П}{омяни`,} гд\си, и= мое` 
недосто'инство, и= прости' ми вся'кое согрjьше'нiе, 
во'льное же и= нево'льное.

\kinovarsimple{И= взима'етъ ча'стицу.}

\kinovarsimple{И= прiе'мъ гу'бу, собира'етъ на дi'скосъ ча^стицы 
до'лjь ст~а'гw хлjь'ба, jа='коже бы'ти во о_у=тверже'нiи, 
и= не и=спа'днути чесому`.}

\kinovarsimple{Та'же дiа'конъ прiе'мъ кади'льницу, и= fv"мiа'мъ 
вложи'въ въ ню`, глаго'летъ къ сщ~е'ннику:}

\kinovar{Б}{л~гослови`,} вл\дко, кади'ло.

\kinovarsimple{И=} \kinovarsimple{а='бiе} \kinovarsimple{са'мъ} \kinovarsimple{глаго'летъ:} \kinovar{Г}{д\су} помо'лимся.

\kinovarsimple{И= сщ~е'нникъ моли'тву кади'ла:}

\kinovar{К}{ади'ло} тебjь` прино'симъ хр\сте` бж~е на'шъ, въ воню` 
благоуха'нiя духо'внагw, _е='же прiе'мъ въ пренб\сный 
тво'й же'ртвенникъ, возниспосли` на'мъ бл~года'ть 
прест~а'гw твоегw` дх~а.

\kinovarsimple{Дiа'конъ:} \kinovar{Г}{д\су} помо'лимся.

\kinovarsimple{Сщ~е'нникъ покади'въ sвjьзди'цу, полага'етъ верху` 
ст~а'гw хлjь'ба, глаго'ля:}

\kinovarsimple{И=} прише'дши sвjьзда` ста` верху`, и=дjь'же бjь` 
_о=троча`.

\kinovarsimple{Дiа'конъ:} \kinovar{Г}{д\су} помо'лимся.

\kinovarsimple{I=ере'й, покади'въ пе'рвый покрове'цъ, покрыва'етъ 
ст~ы'й хлjь'бъ съ дi'скосомъ, глаго'ля:}

\kinovar{Г}{д\сь} воцр~и'ся, въ лjь'поту w=блече'ся: w=блече'ся 
гд\сь въ си'лу, и= препоя'сася. и='бо о_у=тверди` 
вселе'нную, jа='же не подви'жится. гото'въ пр\сто'лъ 
тво'й w\тто'лjь, w\т вjь'ка ты` _е=си`. воздвиго'ша 
рjь'ки гд\си, воздвиго'ша рjь'ки гла'сы своя^. во'змутъ 
рjь'ки сотр_е'нiя своя^, w\т гласw'въ во'дъ мно'гихъ. 
ди'вны высоты^ морскi^я, ди'венъ въ высо'кихъ гд\сь. 
свидjь^нiя твоя^ о_у=вjь'ришася sjьлw`. до'му твоему` 
подоба'етъ ст~ы'ня гд\си, въ долготу` днi'й.

\kinovar{Д}{iа'конъ:} Гд\су помо'лимся. \kinovar{П}{окры'й,} вл\дко.

\kinovarsimple{I=ере'й} \kinovarsimple{же,} \kinovarsimple{покади'въ} \kinovarsimple{вторы'й} \kinovarsimple{покрове'цъ,} \kinovarsimple{покрыва'етъ} \kinovarsimple{ст~ы'й} \kinovarsimple{поти'рь,} \kinovarsimple{глаго'ля:} \kinovar{П}{окры`} нб~са` добродjь'тель 
твоя` хр\сте`, и= хвалы` твоея` и=спо'лнь земля`.

\kinovarsimple{Дiа'конъ:} \kinovar{Г}{д\су} помо'лимся. \kinovar{П}{окры'й,} вл\дко.

\kinovarsimple{I=ере'й же, покади'въ покро'въ, си'рjьчь, возду'хъ, 
покрыва'етъ и= _о=боя^, глаго'ля:}

\kinovar{П}{окры'й} на'съ кро'вомъ крилу` твое'ю и= w\тжени` w\т 
на'съ вся'каго врага` и= супоста'та. о_у=мири` живо'тъ 
на'шъ гд\си, поми'луй на'съ, и= мi'ръ тво'й, и= спаси` 
ду'шы на'шя, jа='кw бл~гъ и= человjьколю'бецъ.

\kinovarsimple{Та'же прiе'мъ сщ~е'нникъ кади'льницу, кади'тъ 
предложе'нiе, глаго'ля три'жды:}

\kinovar{Б}{л~гослове'нъ} бг~ъ на'шъ, си'це бл~говоли'вый, сла'ва 
тебjь`.

\kinovarsimple{Дiа'конъ же на ко'емждо глаго'летъ:}

\kinovar{В}{сегда`,} ны'нjь и= при'снw, и= во вjь'ки вjькw'въ, 
а=ми'нь. \kinovarsimple{И= покланя'ются бл~гоговjь'йнw _о='ба, три'жды.}

\kinovarsimple{Та'же прiе'мъ дiа'конъ кади'льницу, глаго'летъ:}

\kinovarsimple{W=} предложе'нныхъ честны'хъ дарjь'хъ гд\су помо'лимся.

\kinovarsimple{Сщ~е'нникъ же мл~тву предложе'нiя:}

\kinovar{Б}{ж~е} бж~е на'шъ, нб\сный хлjь'бъ, пи'щу всему` мi'ру, 
гд\са на'шего и= бг~а i=и~са хр\ста` посла'вый, сп~са и= 
и=зба'вителя и= бл~годjь'теля, бл~гословя'ща и= 
w=свяща'юща на'съ, са'мъ бл~гослови` предложе'нiе сiе`, 
и= прiими` _е=` въ пренб\сный тво'й же'ртвенникъ. 
помяни`, jа='кw бл~гъ и= чл~вjьколю'бецъ, прине'сшихъ и= 
и='хже ра'ди принесо'ша: и= на'съ неwсужде'ны сохрани` во 
сщ~еннодjь'ствiи бж~е'ственныхъ твои'хъ та^инъ.

\kinovar{Jа='}{кw} ст~и'ся и= просла'вися преч\стно'е и= 
великолjь'пое и='мя твое`, _о=ц~а`, и= сн~а, и= ст~а'гw 
дх~а, ны'нjь и= при'снw, и= во вjь'ки вjькw'въ, а=ми'нь.

\kinovarsimple{И= посе'мъ твори'тъ w\тпу'стъ та'мw, глаго'ля си'це:}

\kinovar{С}{ла'ва} тебjь`, хр\сте` бж~е, о_у=пова'нiе на'ше, 
сла'ва тебjь`.

\kinovarsimple{Дiа'конъ:} \kinovar{С}{ла'ва,} и= ны'нjь: \kinovar{Г}{д\си} поми'луй\kinovarsimple{, три'жды.} 
\kinovar{Б}{л~гослови`.}

\kinovarsimple{Сщ~е'нникъ глаго'летъ w\тпу'стъ.}

\kinovarsimple{А='ще о_у='бw _е='сть недjь'ля:}

\kinovar{В}{оскресы'й} и=з\ъ ме'ртвыхъ хр\сто'съ и='стинный:

\kinovarsimple{А='ще ли же ни`:}

Хр\сто'съ и='стинный бг~ъ на'шъ, моли'твами пречи'стыя 
своея` мт~ре, и='же во ст~ы'хъ _о=ц~а` на'шегw i=wа'нна, 
а=рхiеп\скопа кwнстантiнопо'льскагw, златоу'стагw, \kinovarsimple{[а='ще 
же соверша'ется лiтургi'а вели'кагw васi'лiа, глаго'летъ:} 
васi'лiа, кесарi'и каппадокi'йскiя, вели'кагw:\kinovarsimple{]} и= 
всjь'хъ ст~ы'хъ, поми'луетъ и= спасе'тъ на'съ, jа='кw 
бл~гъ и= человjьколю'бецъ.

\kinovarsimple{Дiа'конъ:} \kinovar{А=}{ми'нь.}

\kinovarsimple{По w\тпу'стjь же кади'тъ дiа'конъ ст~о'е предложе'нiе: 
та'же w\тхо'дитъ, и= кади'тъ ст~у'ю трапе'зу круго'мъ 
крестови'днw, глаго'ля въ себjь`:}

\kinovar{В}{о} гро'бjь пло'тски, \kinovar{в}{о} а='дjь же съ душе'ю jа='кw 
бг~ъ, \kinovar{в}{ъ} раи' же съ разбо'йникомъ, \kinovarsimple{и=} на пр\сто'лjь бы'лъ 
_е=си` хр\сте`, со _о=ц~е'мъ и= дх~омъ, вся^ и=сполня'яй 
неwпи'санный.

\kinovarsimple{_Псало'мъ н~, въ не'мже и= покади'въ ст~и'лище же и= 
хра'мъ ве'сь, вхо'дитъ па'ки во ст~ы'й _о=лта'рь, и= 
покади'въ ст~у'ю трапе'зу па'ки, и= сщ~е'нника, 
кади'льницу о_у='бw w\тлага'етъ на мjь'сто свое`, са'мъ 
же прихо'дитъ ко i=ере'ю.}

\kinovarsimple{И= ста'вше вку'пjь пред\ъ ст~о'ю трапе'зою, 
покланя'ются три'жды, въ себjь` моля'щеся, и= глаго'люще:}

\kinovar{Ц}{р~ю`} нб\сный, о_у=тjь'шителю, ду'ше и='стины, и='же 
вездjь` сы'й и= вся^ и=сполня'яй, сокро'вище благи'хъ и= 
жи'зни пода'телю: прiиди` и= всели'ся въ ны`, и= w=чи'сти 
ны` w\т вся'кiя скве'рны, и= спаси`, бл~же, ду'шы на'шя.

\kinovar{С}{ла'ва} въ вы'шнихъ бг~у, и= на земли` ми'ръ, въ 
чл~вjь'цjьхъ бл~говоле'нiе. \kinovarsimple{Два'жды.}

\kinovar{Г}{д\си,} о_у=стнjь` мои` w\тве'рзеши, и= о_у=ста` моя^ 
возвjьстя'тъ хвалу` твою`.

\kinovarsimple{Та'же цjьлу'ютъ, свяще'нникъ о_у='бw ст~о'е _е=v\глiе, 
дiа'конъ же ст~у'ю трапе'зу. И= посе'мъ подклони'въ 
дiа'конъ свою` главу` свяще'ннику, держя` и= _о=ра'рь 
треми` пе'рсты десны'я руки`, глаго'летъ:}

\kinovar{В}{ре'мя} сотвори'ти гд\сви, вл\дко благослови`.

\kinovarsimple{Свяще'нникъ, зна'менуя _е=го`, глаго'летъ:}

\kinovar{Б}{л~гослове'нъ} бг~ъ на'шъ, всегда`, ны'нjь и= при'снw, 
и= во вjь'ки вjькw'въ.

\kinovarsimple{Та'же} \kinovarsimple{дiа'конъ:} \kinovar{П}{омоли'ся} w= мнjь`, вл\дко.

\kinovarsimple{Сщ~е'нникъ:} \kinovar{Д}{а} и=спра'витъ гд\сь стwпы` твоя^.

\kinovarsimple{И=} \kinovarsimple{па'ки} \kinovarsimple{дiа'конъ:} \kinovar{П}{омяни'} мя, вл\дко ст~ы'й.

\kinovarsimple{Сщ~е'нникъ:} \kinovar{Д}{а} помяне'тъ тя` гд\сь бг~ъ во цр\ствiи 
свое'мъ, всегда`, ны'нjь и= при'снw, и= во вjь'ки 
вjькw'въ.

\kinovarsimple{Дiа'конъ} \kinovarsimple{же} \kinovarsimple{ре'къ:} \kinovar{А=}{ми'нь,} \kinovarsimple{и= поклони'вся, и=схо'дитъ 
сjь'верными две'рьми, поне'же цр\скiя дв_е'ри до вхо'да 
не w\тверза'ются. И= ста'въ на _о=бы'чномъ мjь'стjь, 
пря'мw ст~ы'хъ двере'й, покланя'ется со бл~гоговjь'нiемъ, 
три'жды, глаго'ля въ себjь`:}

\kinovar{Г}{д\си,} о_у=стнjь` мои` w\тве'рзеши, и= о_у=ста` моя^ 
возвjьстя'тъ хвалу` твою`.

\kinovarsimple{И=} \kinovarsimple{посе'мъ} \kinovarsimple{начина'етъ} \kinovarsimple{глаго'лати:} \kinovar{Б}{л~гослови`,} вл\дко.

\kinovarsimple{И=} \kinovarsimple{начина'етъ} \kinovarsimple{свяще'нникъ:} \kinovar{Б}{л~гослове'но} цр\ство:

\kinovarsimple{Вjь'дати} \kinovarsimple{подоба'етъ:} \kinovarsimple{а='ще} \kinovarsimple{без\ъ} \kinovarsimple{дiа'кона} \kinovarsimple{слу'житъ} \kinovarsimple{i=ере'й,} \kinovarsimple{въ} \kinovarsimple{проскомi'дiи} \kinovarsimple{дiа'конскихъ} \kinovarsimple{сло'въ,} \kinovarsimple{и=} \kinovarsimple{на} \kinovarsimple{лiтургi'и} \kinovarsimple{пред\ъ} \kinovarsimple{_е=v\глiемъ,} \kinovarsimple{и=} \kinovarsimple{на} \kinovarsimple{w\твjь'тъ} \kinovarsimple{_е=гw`:} \kinovar{Б}{л~гослови`} вл\дко\kinovarsimple{,} \kinovarsimple{и=} \kinovar{П}{рободи`} вл\дко\kinovarsimple{,} \kinovarsimple{и=} \kinovar{В}{ре'мя} 
сотвори'ти\kinovarsimple{, да не глаго'летъ, то'чiю _е=кт_енiи` и= 
чино'вное предложе'нiе. А='ще же собо'ромъ слу'жатъ 
сщ~е'ннiи мно'зи, дjь'йство проскомi'дiи _е=ди'нъ i=ере'й 
то'кмw да твори'тъ, и= глаго'летъ и=з\ъwбраж_е'нная: 
про'чiи же служи'тели ничто'же проскомi'дiи _о=со'бнw да 
глаго'лютъ.}

